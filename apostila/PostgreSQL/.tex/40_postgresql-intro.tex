\newpage \chapter{Intefaces de Acesso ao PostgreSQL}\setcounter{SteP}{1}
    No jargão de banco de dados, o PostgreSQL utiliza o modelo cliente-servidor.
Uma sessão do PostgreSQL consiste nos seguintes processos (programas) cooperando entre si:

\section{Interfaces de Acesso ao Banco de Dados}\setcounter{SteP}{1}

\begin{itemize}
\item{\bf }O PostgreSQL pode ser acessado a partir de várias linguagens, entre elas
estão:

    - C, C++

    - Java (JDBC)

    - PHP, JSP, ColdFusion

    - TCL/Tk

    - Perl

    - Python

    - ODBC (ASP, Delphi ou qualquer linguagem que suporte ODBC)

\end{itemize}

\section{ Conexão JDBC }\setcounter{SteP}{1}

\begin{itemize}
	\item{\bf }Abaixo um exeomplo de conexção utilizando drive JDBC:
\begin{VerbatimNumerado}
	public Connection connect() {
	    driver = "org.postgresql.Driver";
	    url = "jdbc:postgresql://172.16.128.13:5432/teste?user=
	        postgres";
	    try{
	       Class.forName(drive).newInstance();
	       con = DriverManager.getConnection(url);
	   }
	   catch (Exception e){
	       System.out.println("Error");
	       e.printStackTrace();
	   }
           return con;
	}
\end{VerbatimNumerado}
\item{\bf } Para o URL de conexão temos as opções:

	- {\bf User}: usuário para conexão.

	- {\bf Password}: senha para a conexão.

	- {\bf Database}: nome do banco de dados a se acessado.
     
        - {\bf Port}: porta de conexão.

	- {\bf IP}:  endereço IP do servidor.

\end{itemize}

\section{Configuração ao PostgreSQL}\setcounter{SteP}{1}
Depois de baixar e instalar é hora de configurar. O usuário root do nosso banco de dados é o {\bf postgres}. 
No processo de instalação foi criado um usuário chamdo postgres também no sistema. Então, nos logaremos com este usuário.

\begin{BoxVerbatim}
# Su postgres
\end{BoxVerbatim}
Por padrão, {\bf o usuário de banco de dados ‘postgres’} não tem senha, então agora nos logaremos no shell 
do PostgreSQL para alterar a senha do usuário postgres.

Primeiro, logando no shell…
\begin{BoxVerbatim}
$ psql
\end{BoxVerbatim}
Agora, já no shell do PostgreSQL, vamos alterar a senha do usuário postgres

\begin{BoxVerbatim}
postgres=# ALTER USER postgres WITH PASSWORD 'qualquersenha';
\end{BoxVerbatim}
Esse cara que a gente acabou de configurar ai é o root do banco de dados… 
A gente não vai ficar usando esse usuário nas nossas aplicações, né? Então! vamos criar um novo usuário.

\begin{BoxVerbatim}
postgres=# CREATE USER usuario NOCREATEDB NOSUPERUSER NOCREATEROLE PASSWORD 
	'senha';
\end{BoxVerbatim}
Agora, vamos criar uma tabela também

\begin{BoxVerbatim}
postgres=# CREATE DATABASE minhabase;
\end{BoxVerbatim}


\section{Introdução ao psql}\setcounter{SteP}{1}
\begin{itemize}
\item{\bf }O psql é o modo interativo do PostgreSQL para acesso e manipulação dos bancod de dados.
\begin{BoxVerbatim}
psql [-h hostname -p port -U user -W] [database]
\end{BoxVerbatim}
\item{\bf }Onde:

- {\bf hostname}: nome ou IP do servidor (padrão é localhost)

- {\bf port}: porta de conexão (padrão ŕ 5432)

- {\bf user}: usuário postgresql (padrão é o usuário de sistema operacional)

- {\bf database}: nome do banco de dados.

\item{\bf } A opção -w força a entrada da senha do usuário.
\end{itemize}o


