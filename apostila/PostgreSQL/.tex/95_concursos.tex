\newpage \chapter{Questões de Concurso}\setcounter{SteP}{1}

\section{MPU 2010 (CESPE) - ANALISTA DE INFORMÁTICA / SUPORTE TÉCNICO -
Cargo 27}\setcounter{SteP}{1}

    Julgue os itens subsequentes a respeito dos padrões X.500 e LDAP
(lightweight directory access protocol), usados em serviços de
diretório.

\begin{itemize}
\item{\bf }{\bf 109}: o LDAP não define o serviço de diretório em si, por isso,
no contexto da arquitetura cliente/servidor, o cliente, nesse padrão, é
dependente da implementação do serviço de diretório que está no servidor.

\item{\bf }{\bf 110}: o padrão X.500 especifica um sistema de diretório
distribuído que atende a consultas quanto a objetos da rede. Esse padrão
pode ser utilizado para acessar informações acerca de serviços de hardware,
mas não de software.

\item{\bf }{\bf 111}: no modelo funcional do padrão X.500, o agente do usuário
de diretórios é um processo de aplicação OSI (open system interconnection)
que faz parte do diretório e cuja função é fornecer aos agentes do sistema
de diretório acesso à base de informações.

\item{\bf }{\bf 112}: uma das características do LDAP é que as mensagens do
protocolo de aplicação são transportadas diretamente pela camada TCP
(transport control protocol) da arquitetura da Internet.
\end{itemize}

    Comentários:

\begin{itemize}
\item{\bf }{\bf 109}: o LDAP define o protocolo de acesso as informações do
serviço de diretório (RFC-4511), a base de dados LDAP (RFC-4512) que formam o
serviço de diretório em si. O software cliente é independente da implementação
LDAP do serviço de diretório do servidor.

\item{\bf }{\bf 110}: as informações tipicamente armazenadas em um serviço LDAP
são sobre software e hardware.

\item{\bf }{\bf 111}: o agente do usuário de diretórios (Directory User Agent)
é um software utilizado para acessar as informações armazenadas no agentes
do sistema de diretório (Directory System Agent). O Directory User Agent não
faz parte de Directory System Agent, e fornece ao administrador da rede
acesso à base de informações.

\item{\bf }{\bf 112}: uma das características do LDAP é que as mensagens do
protocolo de aplicação são transportadas pela camada de transporte. Não
existe o conceito de 'camada TCP'.
\end{itemize}
