\newpage \chapter{Liguagem SQL}\setcounter{SteP}{1}

\section{A Linguagem SQL}\setcounter{SteP}{1}
\begin{itemize}
	\item{\bf } SQL (Structured Query Languagem) é uma linguagem declarativa de acesso
	à banco de dados.
	\item{\bf }Por ser uma linguagem padronizada, a migração para o PostgreSQL é 
	fácilitada para aqueles que conhecem SQL.
	\item{\bf }O PostgreSQL está em conformidade com a maior parte das espeficições
	SQL92 e SQL99.
	\item{\bf }A linguagem SQL não considera a caixa dos comandos ({\it xase insensitive}).
	Entretanto, a caixa faz diferença para os leitores entre {\it aspas}.
\end{itemize}

\section{Introdução}\setcounter{SteP}{1}
Este capítulo fornece uma visão geral sobre como utilizar a linguagem SQL para realizar operações simples. 
O propósito deste tutorial é apenas fazer uma introdução e, de forma alguma, ser um tutorial completo sobre a linguagem SQL.
É preciso estar ciente que algumas funcionalidades da linguagem SQL do PostgreSQL são extensões ao padrão.

Conforme criando o usuario e banco de dados, conforme descrito no capítulo anterior, e que o 
psql esteja ativo.

\section{Criação de Tabelas}\setcounter{SteP}{1}
Pode-se criar uma tabela especificando o seu nome juntamente com os nomes das colunas e seus tipos de dado:
\begin{BoxVerbatim}
CREATE TABLE clima {
    ciade           char(80),
    tem_min         int,             -- temperatura mínima
    temp_max        int,             -- temperatura máxima
    prcp            real,            -- precipitação
    data            date     
}
\end{BoxVerbatim}
