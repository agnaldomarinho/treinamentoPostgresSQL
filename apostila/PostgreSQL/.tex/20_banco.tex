\newpage \chapter{Entendendo um Banco de Dados}\setcounter{SteP}{1}

\section{Bancos de Dados Relacionais}\setcounter{SteP}{1}
    
    Um banco de dados é uma aplicação que lhe permite armazenar 
e obter de volta dados com eficiência. O que o torna {\it relacional}
é a maniera como os dados são armazenados e organizados no
banco de dados.

    Quando falamos em banco de dados, aqui, nos referimos a um banco de dados
relacional - RDBMS {\it Relational Database Management System}.

    Em um banco de dados relacional, todos os dados são 
guardados em tabelas. Estas têm uma estrutura que se repete
a cada linha, como você pode observar um uma planilha.
São os relacionamento entre as tabelas que as tornam "relacionais"

\begin{itemize}
\item{\bf }O modelo relacional surgiu devido às seguintes necessidades:
     - Aumentar a independência de dados nos sistemas gerenciadores de bancos de dados;

     - Prover um conjunto de funções apoiadas em álgebra relacional
para armazenamento e recuperação de dados;

\item{\bf } A estrutura fundamental do modelo relacional é a relação.
Uma relação é constituída por um ou mais atributos (campos), que 
traduzem o tipo de dados armazenados. Cada instância do esquema (linha),
designa-se por tupla (registro).
     - O modelo implementa estruturas de dados organizados
em relações (tabelas).
\end{itemize}

\section{Banco de Dados Objeto-Relacional}\setcounter{SteP}{1}

\begin{itemize}
\item{\bf }O PostgreSQL é normalmente considerado um sistema gerenciador de 
banco de dados relacional (SGBD-R, ou RDBMS, em inglês.) Entretanto, 
o PostgreSQL é um sistema gerenciador de banco de dados objeto-relacional (SGBD-OR).

\item{\bf }Por ser objeto-relacional, o PostgreSQL suporta recursos inexistentes
a um banco de dados puramente ralacional, tais como: herança entre
tabelas, arrays em colunas e sobrecarga de funções.
\end{itemize}
