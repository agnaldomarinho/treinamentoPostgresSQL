\newpage \chapter{Introdução ao Postgresql}\setcounter{SteP}{1}

    O PostgreSQL é um SGBD (Sistema Gerenciador de Banco de
Dados) objeto relacional de código aberto, com mais de 15 anos
de desenvolvimento. é extremamente robusto e confiável, além de
ser extremante flexivel e rico em recursos. Ele é considerado 
objeto relacional por implementar, além das características de
um SGBD relacional, algumas características de orientação a
objetos, como herança e tipos personalizados.

\section{O que é o PostgreSQL?}\setcounter{SteP}{1}

\begin{itemize}
\item{\bf } O PostgreSQL é um dos bancos de dados abertos mais utilizados
atualmente, possui recursos avançados e compete igualmente com
muitos bancos de dados comerciais.

\item{\bf } O banco de dados PostgreSQL nasceu na Universidade de Berkeley,
em 1986, como um projeto acadêmico e se encontra hoje na versão 9.1, 
sendo um projeto mantido pela comunidade de {\it Software Livre}.

\item{\bf }A coordernação de desenvolvimento do PostgreSQL é executado pelo 
{\it PostgreSQL Global Development Group} que conta com um grande 
número de desenvolvimento ao redor do mundo.
\end{itemize}
