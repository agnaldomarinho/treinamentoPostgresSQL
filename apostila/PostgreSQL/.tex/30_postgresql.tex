:\newpage \chapter{Introdução ao Postgresql}\setcounter{SteP}{1}

    O PostgreSQL é um SGBD (Sistema Gerenciador de Banco de
Dados) objeto relacional de código aberto, com mais de 15 anos
de desenvolvimento. é extremamente robusto e confiável, além de
ser extremante flexivel e rico em recursos. Ele é considerado 
objeto relacional por implementar, além das características de
um SGBD relacional, algumas características de orientação a
objetos, como herança e tipos personalizados.

\section{O que é o PostgreSQL?}\setcounter{SteP}{1}

\begin{itemize}
\item{\bf } O PostgreSQL é um dos bancos de dados abertos mais utilizados
atualmente, possui recursos avançados e compete igualmente com
muitos bancos de dados comerciais.

\item{\bf } O banco de dados PostgreSQL nasceu na Universidade de Berkeley,
em 1986, como um projeto acadêmico e se encontra hoje na versão 9.1, 
sendo um projeto mantido pela comunidade de {\it Software Livre}.

\item{\bf }A coordernação de desenvolvimento do PostgreSQL é executado pelo 
{\it PostgreSQL Global Development Group} que conta com um grande 
número de desenvolvimento ao redor do mundo.

\item{\bf } Ele é um SGBD muito adequado para o estudo universitário do 
modelo relacional, além de ser uma ótima opção para empresas 
implatarem soluções de alta confiabilidade sem altos custos de 
licenciamento.

\item{\bf } É um programa distribuido sob a licença BSD, o que 
torna o seu código fonte disponível e o seu uso livre para aplicações
comercias ou não.

\item{\bf } O PostgreSQL foi implementado em diversos ambientes de produção 
no mundo, entre eles, um bom exemplo do seu pontencial é o banco de dados
que armazena os registro de domínio .org, mantido pela empresa
	Afilias.
\end{itemize}

\section{ Principais Funcionalidades}\setcounter{SteP}{1}

\begin{itemize}
\item{\bf }Banco de dados objeto-relacional

    - Herança entre as tabelas

    - Sobrecarga de funções

    - Colunas do tipo array

\item{\bf }Suporte a transações (padrão ACID)

\item{\bf }Lock por registro (row level locking)

\item{\bf }Integridade referencial

\item{\bf }Sub-consultas.

\item{\bf }Controle de concorrência multi-versão (MVCC);

\item{\bf }Funções armazenadas (Stored Procedures), que podem ser escritas em várias linguagens de programação (PL/PgSQL, Perl, Python, Ruby, e outras);

\item{\bf }Gatilhos (Triggers);

\item{\bf }Tipos definidos pelo usuário;

\item{\bf }Esquemas (Schemas);

\item{\bf }Conexões SSL.

\item{\bf }Áreas de armazenamento (Tablespaces)

\item{\bf }Pontos de salvamento (Savepoints)

\item{\bf }Commit em duas fases

\item{\bf }Arquivamento e restauração do banco a partir de logs de transação

\item{\bf }Diversas ferramentas de replicação

\item{\bf }Extensões para dados geoespaciais, indexação de textos, xml e várias outras.

\item{\bf }Acesso via drivers ODBC e JDBC, além do suporte nativo em várias linguagens

\item{\bf }Suporte ao armazenamento de BLOBs (binary large objects)

\item{\bf }Sub-queries e queries na cláusula FROM

\item{\bf }Sofisticado mecanismo de tuning

\item{\bf }Suporte a conexão de banco de dados seguras (criptografia)

\item{\bf }Modelo de segurança para o acesso aos objetos do banco de dados por roles

\item{\bf }Triggers views e functions (PL/pgSQL, Perl, Python e Tcl

\item{\bf }Mecanismos próprio de logs
\end{itemize}

\section{Plataforma Suportadas}\setcounter{SteP}{1}

\begin{itemize}
\item{\bf }IBM AIX
\item{\bf }FreeBSD$<$ OpenBSD, NetBSD
\item{\bf }HP-UX
\item{\bf }Irix
\item{\bf }Linux
\item{\bf }MarcOS X
\item{\bf }Microsoft Windows (suporte nativo desde a versão 8.0)
\item{\bf }SCO Open Server
\item{\bf }Sun Solaris
\item{\bf }Tru64 Unix
\item{\bf }Unix Ware
\end{itemize}

