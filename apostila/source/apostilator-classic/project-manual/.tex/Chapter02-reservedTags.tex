\newpage \chapter{Caracteres Reservados}\setcounter{SteP}{1}
\label{chap02}

Ao elaborar um manual relacionado à linguagem HTML por exemplo, é possível que
seja necessário utilizar alguma {\it tag} reservada do apostilator o que
causaria um efeito indesejado. Para contornar essa dificuldade foram
adicionados alguns 'marcadores' reservados para a elaboração de {\it tags}.
São elas:

\begin{table}[H]
  \begin{center}
  \begin{tabular}{|c|c|}
    \hline {\bf Caracteres} 	&	 {\bf Marcadores}  \\ \hline \hline
     $<$ 	&	 ;lt;  \\ \hline
     $>$ 	&	 ;gt;  \\ \hline
     \& 	&	 ;amp;  \\ \hline
\end{tabular}
    \caption{Caracteres especiais}
    \label{tab:specialChars}
  \end{center}
\end{table}

\begin{BoxVerbatim}
Veja só como fica fácil falar da tag <tr>, usando os 
marcadores especiais! O problema é que até o momento só 
funciona dentro de uma ambiente de <comando>.
\end{BoxVerbatim}
