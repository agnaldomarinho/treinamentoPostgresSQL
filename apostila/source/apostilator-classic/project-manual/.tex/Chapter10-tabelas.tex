\newpage \chapter{Construindo tabelas}\setcounter{SteP}{1}
\label{sec:tabelas}

Os itens mais complexos a serem inseridos em um documento utilizando essa linguagem
são as tabelas.

\begin{BoxVerbatim}
    <tabela>{|l|c|r|}
        <lh><th>Col 1</th> <col> <th>Col 2</th> <col> <th>Col 3</th></lh>
        <tr> à esquerda <col> centralizado <col> à direita </tr>
        <tr> texto      <col> mais texto   <col> muito texto </tr>
        <tr> l = left   <col> c = center   <col> r = right   </tr>
    </tabela>
\end{BoxVerbatim}

O resultado dessa tabela é o seguinte:

    \begin{table}[H]
  \begin{center}
  \begin{tabular}{|l|c|r|}
        \hline{\bf Col 1} 	&	 {\bf Col 2} 	&	 {\bf Col 3} \\ \hline \hline
         à esquerda 	&	 centralizado 	&	 à direita  \\ \hline
         texto      	&	 mais texto   	&	 muito texto  \\ \hline
         l = left   	&	 c = center   	&	 r = right    \\ \hline
\end{tabular}
        \caption{tabela exemplo}
        \label{tab:exemplo}
      \end{center}
\end{table}

