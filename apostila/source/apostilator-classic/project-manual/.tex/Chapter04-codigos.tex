\newpage \chapter{Inserir Comandos / Linhas de Código}\setcounter{SteP}{1}
\label{xxx}

Para inserir linhas de código, comandos ou arquivos de configuração, é necessário utilizar o par de {\it tags} como no exemplo a seguir:

\begin{BoxVerbatim}
    <comando>
           Qualquer tag inserida dentro das tags: <comando></comando>
       não serão interpretadas pelo processador de texto. Assim como não há quebra-linha
       automática (como nesta linha).
           Dessa forma, se você adicionar 5 espaços como no exemplo abaixo,
      eles serão impressos:
           Dentro do parênteses há 5 (     ) espaços! :o)
    </comando>
\end{BoxVerbatim}

    Repare que o texto acima foi colocado dentro de um ambiente para comandos. Note que todos os espaços em branco foram impressos exatamente da forma que eu os adicionei! Além disso a fonte dentro de um ambiente comando é alterada para destacar-se do texto comum. Dentro desse ambiente pode ser inserido qualquer coisa, inclusive tags.

 Existe também a tag {\bf comandoNumerado} que adiciona o número da linha
 dentro de um ambiente de comandos. Veja o exemplo:

\begin{VerbatimNumerado}
Para utilizar essa tag, basta fazer o seguinte:

<comandoNumerado>
Comandos....
Mais comandos...

muitos comandos....

Chega de comandos!!!
</comandoNumerado>
\end{VerbatimNumerado}
