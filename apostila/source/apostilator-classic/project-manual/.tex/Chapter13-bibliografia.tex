\newpage \chapter{Adicionando Bibliografia}\setcounter{SteP}{1}

Para adicionar uma referência bibliográfica em seu texto, você deve adicionar
uma entrada para ela no arquivo {\bf Apostila/bibliografia.bib}. Cada entrada
de bibliografia deve seguir o formato {\bf exato} mostrado abaixo:

\begin{BoxVerbatim}
@Article{tcpipIBM,
    title   =   "Configuring {TCP}/{IP} under Linux" ,
    author  =   "Tom Syroid" ,
    journal =   "{IBM} developerWorks" ,
    note    =   "http://www.ibm.com/developerWorks" ,
    year    =   2001
}

@Misc{cansian,
    title   =   "Tutorial Iptables - firewalls" ,
    author  =   "Adriano Cansian et al" ,
    year    =   2003 
}
\end{BoxVerbatim}

Neste exemplo há duas entradas bibliográficas que podem ser referenciadas no
meio do texto por seus "apelidos", ou seja, {\bf tcpipIBM} e {\bf cansian},
utilizando a tag "citar" como mostrado no exemplo a seguir:

\begin{BoxVerbatim}
 Este aqui é um treco de um texto contendo uma referência
 bibliográfica <citar>tcpipIBM</citar>.
\end{BoxVerbatim}

Ainda quanto as definições de referências bibliográficas, veja que na primeira,
{\bf tcpipIBM} há os campos {\bf note} e {\bf journal} que não estão
presentes na segunda referência {\bf cansian}. Sendo assim, quando você não
utilizar todos os campos, basta removê-los e tomar cuidado com as vírgulas!
