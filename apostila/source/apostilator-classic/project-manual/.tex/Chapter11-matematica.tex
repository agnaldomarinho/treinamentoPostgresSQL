\newpage \chapter{Expressões Matemáticas e Equações}\setcounter{SteP}{1}
\label{sec:equacao}

Para adicionar equações ou expressões matemáticas a um texto devemos informar
ao processador de texto que estamos entrando em um ambiente matemático. Se a
expressão a ser entrada deve estar misturada ao texto como em
$2^3=8$, então utilizamos a {\it tag} 'math', agora, se desejarmos
entrar com uma equação que requer uma linha inteira só pra ela como a da Lei de
Gauss elétrica, por exemplo, que é descrita por

\begin{equation}
\int\int_A \vec E.d\vec a = \int \int \int_V \nabla . \vec E dV = \frac{\rho}{\epsilon_0}
\label{eq:gauss}
\end{equation}

Para inserir a expressão matemática $2^3=8$ o código utilizado foi:

\begin{BoxVerbatim}
"...misturada ao texto como em <math>2^3=8</math>, então utilizamos..."
\end{BoxVerbatim}

O código utilizado para inserir a equação~\ref{eq:gauss} é o seguinte:

\begin{BoxVerbatim}
<equacao>
\int\int_A \vec E.d\vec a = \int \int \int_V \nabla . \vec E dV 
= \frac{\rho}{\epsilon_0}
<nome>eq:gauss</nome>
</equacao>
\end{BoxVerbatim}

Note que a {\it tag} 'name' foi adicionada ao ambiente matemático para que
fosse possível referir-se à fórmula pelo seu número.

A sintaxe utilizada para elaboração de expressões matemáticas deve ser a do
processador de texto \LaTeX2e e pode ser encontrada facilmente no livro:
"The not so short introduction to Latex" (google: lshort filetype:pdf).
