\newpage \chapter{Inserindo Figuras}\setcounter{SteP}{1}
\label{sec:figuras}

A inserção de figuras ao documento é bastante simples. O autor deverá
especificar alguns parâmetros da figura, como o nome do arquivo, legenda e
tamanho que o processador posicionará e atribuirá um número à figura
automaticamente. 

Outro atributo importante é o 'nome' da figura pois, com ele será
possível fazer referências a ela por intermédio da {\it tag} 'ref'. A sintaxe
básica para a inserção de uma figura é a seguinte:

\begin{BoxVerbatim}
    <figura>
        <tamanho>0.2</tamanho>
        <arquivo>imgs/nuclear_marvin</arquivo>
        <legenda>Legenda da figura</legenda>
        <nome>fig:marvin</nome>
    </figura>
\end{BoxVerbatim}

A imagem que será inclusa pode ser vista na fig.~\ref{fig:marvin}.

\begin{figure}[H]\begin{center}
    \includegraphics[width=0.2\columnwidth]
    {imgs/nuclear_marvin}
    \caption{Legenda da figura}
    \label{fig:marvin}
\end{center}\end{figure}


O significado de cada atributo está especificado na lista a seguir:

\begin{enumerate}
	\item{\bf tamanho} este atributo especifica a largura da figura em
	relação à largura da página. Os valores válidos para o tamanho de uma
	figura são os compreendidos na faixa de 0 a 1. Dessa forma, podemos
	considerar que o tamanho será entre 0 e 100\% da largura da folha. 
	Com isso, especificando uma figura cujo tamanho é 0.6, estaremos definindo 
	que a largura dessa figura será 60\% do tamanho da largura da página;

	\item{\bf legenda} este atributo define a legenda da figura. Deve ser um
	texto breve explicando o significado da figura.

	\item{\bf arquivo} neste atributo está especificado o nome da figura e
	seu path. Note que todas as figuras {\bf devem} estar no diretório 'imgs'
	dentro do diretório 'Apostila' (\~{}/apostilator/Apostila/imgs). Entretanto,
	o parâmetro a ser passado para a {\it tag} 'arquivo' é apenas
	'imgs/nome\_figura'. Não é necessário especificar o formato da figura.
	
	\item{\bf nome} a {\it tag} 'nome' é a responsável por atribuir um
	apelido para a figura tornando possível referenciá-la através da tag 'ref'
	(veja sec.~\ref{sec:referencia})

\end{enumerate}

Todas as figuras serão automaticamente centralizadas na página. Os formatos
de imagem aceitos pelo processador de texto são: gif, jpg, png dentre outros.
Só não são aceitos 'ps' e 'eps'.

Note que, no caso de figuras, a ordem das tags {\bf não} pode ser alterada!!
