\newpage \chapter{Listas e Listas Numeradas}\setcounter{SteP}{1}
\label{sec:listas}

\section{Listas não numeradas}\setcounter{SteP}{1}

Uma forma comum de apresentar tópicos de um determinado assunto é através de
de listas. Há basicamente dois tipos de listas as que apenas destacam cada item
e aquelas que numeram os itens.

Para criar uma lista simples, sem numeração, procedemos da seguinte forma:

\begin{BoxVerbatim}
    <lista>
        <item>propriedade 1</item> este item tem como título o trecho 
            entre as tags com a palavra 'item', ou seja 'propriendade 1';
        <item>propriedade 2</item> note que o título desse item é 
            'propriedade 2' e que na frente de todo os itens há uma 
            bolinha preta afim de dar um destaque;
        <item>propriedade 3</item> vale ressaltar que o título de cada
            item, ou seja, propriedade 1, 2 e 3, serão destacados do resto 
            do texto pois ficarão em negrito automaticamente.
    </lista>
\end{BoxVerbatim}

O efeito do trecho de código acima será o seguinte:

\begin{itemize}
	\item{\bf propriedade 1} este item tem como título o trecho entre as tags
							com a palavra 'item', ou seja 'propriendade 1';
	\item{\bf propriedade 2} note que o título desse item é 'propriedade 2' e
							que na frente de todo os itens há uma bolinha
							preta afim de dar um destaque;
	\item{\bf propriedade 3} vale ressaltar que o título de cada item, ou
							seja, propriedade 1, 2 e 3, serão destacados do
							resto do texto pois ficarão em negrito
							automaticamente.
\end{itemize}

A lista acima apenas apresenta os itens. Se desejarmos enumerar cada item, basta
alterar a {\it tag} inicial de 'lista' para 'enumerar' como mostrado a
seguir:

\section{Listas Numeradas}\setcounter{SteP}{1}

\begin{BoxVerbatim}
    <enumerar>
        <item>propriedade</item> este item tem como título o trecho entre
            as tags com a palavra 'item', ou seja 'propriendade';
        <item>propriedade</item> note que o título desse item também é 
            'propriedade' e que na frente de todo os itens há agora uma 
            numeração de acordo com a ordem de entrada dos itens;
        <item>propriedade</item> vale ressaltar que o título de cada item, 
            ou seja, propriedade 1, 2 e 3, serão destacados do resto do 
            texto pois ficarão em negrito automaticamente.
    </enumerar>
\end{BoxVerbatim}

O resultado do texto acima é o seguinte:

\begin{enumerate}
	\item{\bf propriedade} este item tem como título o trecho entre as tags
							com a palavra 'item', ou seja 'propriendade';
	\item{\bf propriedade} note que o título desse item também é
							'propriedade' e que na frente de todo os itens há
							agora uma numeração de acordo com a ordem de
							entrada dos itens;
	\item{\bf propriedade} vale ressaltar que o título de cada item, ou
							seja, propriedade 1, 2 e 3, serão destacados do
							resto do texto pois ficarão em negrito
							automaticamente.
\end{enumerate}


Sendo assim, a forma de trabalhar com listas é a mesma, não importa se é uma
lista numerada ou não. A única diferença entre elas é a {\it tag} principal,
ou seja 'lista' ou 'enumerar'.
