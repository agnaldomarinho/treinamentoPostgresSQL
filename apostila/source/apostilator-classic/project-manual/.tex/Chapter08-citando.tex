\newpage \chapter{Citando tabelas, figuras ou capítulos}\setcounter{SteP}{1}
\label{sec:citando}

Um dos atributos de uma tabela ou de uma figura (sendo ela um gráfico, um
diagrama ou mesmo um desenho) é a sua numeração. Em um texto, todas as tabelas
e figuras devem ser numerados e quando nos referimos a elas devemos fazer
referência a esse número. Entretanto não precisamos nos preocupar com o
número dos capítulos, subseções, tabelas ou figuras, o próprio software
controla a numeração. 

Sendo assim, a forma mais prática de fazer referência a
uma tabela, figura, capítulo ou subseção é definindo um nome para o objeto
desejado e depois fazer referência a esse nome, por exemplo: se eu desejo citar
o número da seção que está sendo explicado como trabalhar com listas, basta eu
dizer que ela é a seção~\ref{sec:listas}. Não, eu não inclui o número dessa
seção manualmente, mas sim da seguinte forma:

Quando inicio um capítulo eu adiciono o seguinte conjunto de {\it tags}:

\begin{BoxVerbatim}
<capitulo>Como fazer referência a um capítulo ou tabela ou...</capitulo>
\end{BoxVerbatim}

Mas, para que eu possa citar esse capítulo pelo seu número eu tenho que
adicionar um outro conjunto adicional de {\it tags} atribuindo um nome à esse 
capítulo. Isso se faz utilizando as {\it tags} 'nome' da seguinte forma:

\begin{BoxVerbatim}
<capitulo>Como fazer referência a um capítulo ou tabela ou...</capitulo>
<nome>sec:referencia</nome>
\end{BoxVerbatim}

Dessa forma, atribuímos o nome 'sec:referencia' ao capítulo 'Como fazer
referência a ...'.

Uma vez atribuído um nome ao capítulo ou subseção basta fazer referência ao
nome dele da seguinte forma:

\begin{BoxVerbatim}
Estou fazendo referência ao capítulo<ref>sec:referencia</ref>
\end{BoxVerbatim}

O nome entre as {\it tags} 'ref' será substituído pelo numero da seção que
possui como nome 'sec:referencia'

O princípio para adicionar nomes a tabelas, figuras e equações é o mesmo e
será mostrado nas seções~\ref{sec:tabelas}, ~\ref{sec:figuras} e
~\ref{sec:equacao} respectivamente.  Entretanto a forma de fazer
referência a elas é a mesma!

Para citar uma bibliografia use:
\begin{BoxVerbatim}
Conforme o TANENBAUM <citar>Test1</citar>, etc...
\end{BoxVerbatim}

Gerando: Conforme o TANENBAUM ~\cite{Test1}, etc...

