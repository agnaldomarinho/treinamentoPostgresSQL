\newpage \chapter{Capítulos: Chapters.lst}\setcounter{SteP}{1}

    Ao redigir um texto muito grande é interessante dividí-lo em arquivos menores, facilitando a manipulação e tornando o processo mais eficiente.

    Recomanda-se que seja criado um arquivo por capítulo, ao qual deve ser inserido no arquivo 'Chapters.lst' que contém a lista ordenada dos capítulo do projeto.

\section{Sintaxe do 'Chapters.lst'}\setcounter{SteP}{1}

    A sintaxe do arquivo {\bf Chapters.lst}:

\begin{itemize}
  \item{\bf Comentários:} o caracter '\#' faz com que a linha por ele iniciada seja considerada como comentário;
  \item{\bf Um arquivo por linha:} cada linha do arquivo corresponde a apenas um arquivo;
  \item{\bf Sem extensão .xml:} apenas o nome do arquivo, sem extensão;
  \item{\bf Ordenados:} a ordem na qual os arquivos são inseridos será a mesma ordem que o processador texto irá inseri-los no documento.
\end{itemize}
